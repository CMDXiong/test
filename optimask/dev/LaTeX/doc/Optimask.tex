 % LaTeX (version 2e) source for Optimask Documentation
 % IF USE CHINESE Language PACKAGES, THEN MUST USE XeLaTeX to COMPILE !!!
 % Ref.: \OptiXera\Develop\Docs\LaTeX, or \OptiXera\Develop\optixera\optimask\dev\LaTeX\
 % by hdeng@optixera.com
% Created: 2016-12-17; Updated: 2017-05-15;

%======================================================================
%   P R E A M B L E
%======================================================================

%%THIS for pdf Latex with .pdf figures%%%%%%%%%%%%%%

% memoir style %Books split into Volumns
%\documentclass[12pt,twoside]{book}
\documentclass[10pt,twoside]{book}
%\documentclass[journal,twoside]{IEEEtran}

%\documentclass[pdftex]{report}
\usepackage{ifpdf}
\ifpdf
\usepackage[pdftex]{graphicx}
\else
\usepackage[dvips]{graphicx}
\fi
%%%%%%%%%%%%%%%%%%%%%%%%%%%%%%%%%%%%%%%%%%%%%%%

%\usepackage{watermark}; %\usepackage{everypage}; %\usepackage{draftwatermark}; %\usepackage{ncccropmark}

\usepackage{fancyhdr} %fancy headings
\usepackage{amsmath}
\usepackage{cite}
\usepackage{notoccite} %Prevent erroneous numbering of cites when using BibTeX/unsrt
%\usepackage[dvips=true,bookmarks=true]{hyperref} %only needed for PDF generation
%\usepackage{makeidx} %if need an index
\usepackage[printonlyused]{acronym} %\usepackage{acronym} %if need list of acronyms
%\usepackage{nomencl}
%\usepackage{stfloats}
%\usepackage{float}
\usepackage{subfigure}
%\usepackage{array}
%\usepackage{psfrag}
\usepackage{color}
\usepackage{listings}
%\usepackage[normalem]{ulem}  %to underline serveral lines; NO underline for \emph{}!!

%Custom packages:
%\usepackage[first]{datestamp} %first, all, off.
\usepackage[all]{datestamp}
%Datestamp on first page of each chapter
%\usepackage{glosstex} %preparation of glossaries, lists of acronyms or sorted lists in general

% *** CHINESE Language PACKAGES (Select One of These Methods)***
%% !! MUST USE XeLaTeX to COMPILE !!! OTHERWISE, WILL HAVE ERROR!
%
%% Method 1: Refer to testXeTeX.tex, use XeLaTex+fontspec package
%\usepackage{fontspec, xunicode, xltxtra}
%\XeTeXlinebreaklocale "zh"
%\XeTeXlinebreakskip = 0pt plus 1pt minus 0.1pt
%\setmainfont{SimSun} %新宋体 %{FZShuTi}字体不存在,用{FZShuTi}字体编译出错。
% 
%\setmainfont{Times}
%
%% Method 2: Refer to testXeCJK.tex, use XeLaTex+XeCJK package
\usepackage{xeCJK} % 调用 xeCJK 宏包
\setCJKmainfont{SimSun} % 设置 CJK 主字体为 SimSun (宋体)
%TeX default font: Computer Modern Roman, size: 10pt.

%----------------------------------------------------------------------
%\makeindex % activate index-making
%\makeglossary %activate nomencl (list of symbols) making

%----------------------------------------------------------------------
% Set "1.2" as the line spacing throughout the thesis for readability (optional).
\renewcommand{\baselinestretch}{1.2}

%----------------------------------------------------------------------
% Reset page margins properly (commented is for doublesided pages)
\setlength{\marginparwidth}{0pt}
\setlength{\marginparsep}{0pt}
\setlength{\oddsidemargin}{0.125in}
\setlength{\evensidemargin}{0.125in}
\setlength{\textwidth}{6.375in}
\raggedbottom
%----------------------------------------------------------------------
%My own command & environment definitions (More in course.tex of Waterloo's source.zip):
%Some Latin abbreviations in italic (\MakeLowercase{\textit{et al.}}; space after "\" produces a space):
\newcommand{\eg}{\textit{e.g.}}      %e.g. = exempli gratia = for example
\newcommand{\ie}{\textit{i.e.}}      %i.e. = id est = that is
\newcommand{\cf}{\textit{c.f.}}      %c.f. = confer = compare
\newcommand{\etc}{\textit{etc}.\@}    %etc = et cetera = and the rest
\newcommand{\etal}{\textit{et~al}.\@} %et al = et alia = and other
%\newcommand{\degree}{^\circ}


%%Define own words
 \def\InP{$\mathrm{InP}$}
 \def\InGaAsP{$\mathrm{InGaAsP}$}
 \def\InGaAsPxy{$\mathrm{In_{1-x}Ga_{x}As_{y}P_{1-y}}$}
 \def\um{$\mu\mathrm{m}$}
 \def\nm{$\mathrm{nm}$}

%----------------------------------------------------------------------
%%%to include subsubsections in the table of contents
%%%%%\renewcommand*\thesubsubsection{\Roman{subsubsection}.} %\arabic{},\alph{},\roman{},\Alph{},\Roman{}
\setcounter{secnumdepth}{3} %%add numbering depth
\setcounter{tocdepth}{3}

%======================================================================
%   L O G I C A L    D O C U M E N T
%======================================================================
\begin{document}

%----------------------------------------------------------------------
% FRONT MATERIAL
%----------------------------------------------------------------------
% TITLE PAGEs (may need to space and format this page yourself)
% By Henghua DENG; Created: 2009-11-20; Updated: 2009-11-20;

\pagestyle{empty} % No headers or page numbers

\begin{center}

\vspace*{1.0cm} \Huge{Optimask Manual}

\vspace*{1.0cm}
\normalsize
by \\
\vspace*{1.0cm}
\Large
Optixera\\
\vspace*{2.0cm}
\normalsize
Optixera \\
USA and China\\


\vspace*{2.0cm} Silicon Valley, California, USA, \the\year\\ %\the\year, %\today
%\vspace*{2.0cm} Silicon Valley, California, USA, 2012\\

\vspace*{2.0cm} \copyright Optixera \the\year\\
\end{center}

\newpage

%%%%%%%%%%%%%%%%%%%%%%%%%%%%%%%%%%%%%%%%%%%%%%%%%%%%%%%%%%%%%
% PRELIMINARY PAGES
%%%%%%%%%%%%%%%%%%%%%%%%%%%%%%%%%%%%%%%%%%%%%%%%%%%%%%%%%%%%%
\pagestyle{plain} % No headers, just page numbers
\pagenumbering{roman} % Roman numerals
\setcounter{page}{2}

%% Declaration Page for ELECTRONIC SUBMISSION OF A THESIS
%\begin{center}\large \textbf{Author's Declaration For Electronic Submission} \end{center}
%\noindent I hereby declare that I am the sole author of this report. \\

%\noindent I understand that my thesis may be made electronically
%available to the public. \vspace{4cm}

%\noindent Henghua Deng
%\newpage

% Long abstract (manually formatted)
\resetdatestamp %Date Stamp--Only use when custom package datestamp.sty is used.
\begin{center}\Large \textbf{Abstract} \end{center}
\addcontentsline{toc}{chapter}{Abstract}
Optimask is a revolutionary layout tool for advanced photomask drawing. It is developed after decades of professional experience from elite engineers and researchers in semiconductor and photonics industry companies, as well as scientific institutes and universities. Our team of international developers consist of photonic experts, electronics researchers and computer scientists. Our primary and ultimate goal is to simplify photonics simulation and photomask design as easy as word processing, with increased accuracy, precision, versatility, and portability. We will make photomask drawing a pleasure, and remove the barrier of advanced knowledge, skill-set and training required.

\vspace*{1em} %\textbf{Index Terms:}
\begin{description}
    \item[\emph{Index Terms}:]
      \small{\acf{GDSII}}
\end{description}
\newpage

% Acknowledgements and/or Dedication Pages
\begin{center}\Large \textbf{Acknowledgements}\end{center}
\addcontentsline{toc}{chapter}{Acknowledgements}

We would like to express our sincere gratitude to all people who helped with this product.

\begin{flushright}
Optixera \\ 10/22/2007--\today \\
Revised: \today \\
\end{flushright}

\newpage

% Pages which are generated automatically
%%\setcounter{page}{6} % Set this counter to get correct page numbers
\tableofcontents
\listoftables
\addcontentsline{toc}{chapter}{List of Tables}
\listoffigures
\addcontentsline{toc}{chapter}{List of Figures}
\newpage

% Pages of List of Acronyms

\chapter*{List of Acronyms}
%Chapter style and no numbering, not included in the Table of Contents (TOC)
%For smaller fonts: \begin{center} \Large \textbf{List of Acronyms} \end{center}

%\ac{}--General Form. To enter an acronym inside the text.
%       The first time you use an acronym, the full name of the acronym along
%       with the acronym in brackets will be printed.
%\acf{}--To get Full Name (full version) of the acronym.
%\acs{}--To get the short version of the acronym.
%\acl{}--To get the expanded acronym without mentioning the acronym
%\acp{}, \acfp{}, \acsp{}, aclp{}--plural form of \ac{}, \acf{}, \acs{}, acl{}
%
\begin{acronym}[ATGONUPIC]
 \acro{2D}{two-dimensional}
 \acro{3D}{three-dimensional}
 \acro{AES}{Auger Electron Spectroscopy}
 \acro{AFM}{Atomic Force Microscopy}
 \acro{AON}{All Optical Network}
 \acro{APD}{Avalanche Photodiode}
 \acro{ASIC}{Application Specific Integrated Circuit}
 \acro{ATG}{Asymmetric Twin-Waveguide}
 \acro{ATR}{Attenuated Total Reflection}
 \acro{AWG}{Arrayed Waveguide Grating}
 \acro{BPM}{Beam Propagation Method}
 \acro{BPD}{Balanced Photo-Detectors}
 \acro{CMT}{Coupled Mode Theory}
 \acro{CMOS}{Complementary Metal Oxide Semiconductor}
 \acro{CROW}{Coupled Resonator Optical Waveguide}
 \acro{CVD}{Chemical Vapor Deposition}
 \acro{dB}{decibel}
 \acro{DBR}{Distributed Bragg Reflector}
 \acro{DFB}{Distributed-Feedback}
 \acro{d.f.}{Degrees of Freedom}
 \acro{DQPSK}{Differential Quadrature Phase Shift Keying}
 \acro{DSP}{Digital Signal Processing}
 \acro{DUT}{Device Under Test}
 \acro{DVT}{Design Verification Test}
 \acro{DWDM}{Dense Wavelength Division Multiplexing}
 \acro{EA}{Evolutionary Algorithm}
 \acro{EC}{Evolutionary Computation}
 \acro{EDS}{Energy dispersive x-ray spectroscopy}
 \acro{EIM}{Effective Index Method}
 \acro{EMT}{Effective Medium Theory}
 \acro{EPON}{Ethernet PON}
 \acro{ER}{Extinction Ratio}
 \acro{FD}{Finite-Difference}
 \acro{FDM}{Finite Difference Method}
 \acro{FDTD}{Finite-Difference Time-Domain method}
 \acro{FDTD-BPM}{Finite-Difference Time-Domain Beam-Propagation Method}
 \acro{FD-BPM}{Finite-Difference Beam-Propagation Method}
 \acro{FE}{Finite-Element}
 \acro{FEA}{Finite Element Analysis}
 \acro{FEM}{Finite Element Method}
 \acro{FETD-BPM}{Finite-Element Time-Domain Beam-Propagation Method}
 \acro{FE-BPM}{Finite-Element Beam-Propagation Method}
 \acro{FFT-BPM}{Fast-Fourier-Transform Beam-Propagation Method}
 \acro{FP}{Fabry-Perot}
 \acro{FTTH}{Fiber-To-The-Home}
 \acro{FTTP}{Fiber-To-The-Premises}
 \acro{FV}{Full Vectorial}
 \acro{FWHM}{Full Width at Half Maximum}
 \acro{GA}{Genetic Algorithm}
 \acro{GDSII}{Graphic Data System II}
 \acro{GPON}{Gigabit PON}
 \acro{ID-BPM}{Imaginary-Distance Beam-Propagation Method}
 \acro{IC}{Integrated Circuit}
 \acro{ITU-T}{International Telecommunications Union Telecommunication Standardization Sector}
 \acro{MBE}{Molecular Beam Epitaxy}
 \acro{MEMS}{Micro Electro Mechanical System}
 \acro{MFD}{Mode Field Diameter}
 \acro{MGVI}{Multi-Grid Vertical Integration}
 \acro{MMI}{Multi-Mode Interferometer}
 \acro{MQW}{Multiple Quantum Well}
 \acro{MZDI}{Mach-Zehnder Delay Interferometer}
 \acro{MZI}{Mach-Zehnder Interferometer}
 \acro{MZM}{Mach-Zehnder Modulator}
 \acro{OADM}{Optical Add-Drop Multiplexer}
 \acro{OEIC}{Optoelectronic Integrated Circuit}
 \acro{OLT}{Optical Line Terminal}
 \acro{OMT}{Optical Modal Transformer}
 \acro{ONU}{Optical Network Unit}
 \acro{OPAD}{Optically Pre-Amplified Detector}
 \acro{PARC}{Passive Active Resonant Coupler}
 \acro{PBG}{Photonic Bandgap device}
 \acro{PhC}{Photonic Crystal}
 \acro{PCF}{Photonic Crystal Fiber}
 \acro{PCW}{Photonic Crystal Waveguide}
 \acro{PCM}{Process Control Monitor}
 \acro{PECVD}{Plasma Enhanced Chemical Vapor Deposition}
 \acro{PD}{Photodetector}
 \acro{PDL}{Polarization Dependent Loss}
 \acro{PIC}{Photonic Integrated Circuit}
 \acro{PLC}{Planar Lightwave Circuit}
 \acro{PMD}{Polarization Mode Dispersion}
 \acro{PML}{Perfectly Matched Layer}
 \acro{PON}{Passive Optical Network}
 \acro{QAM}{Quadrature Amplitude Modulation}
 \acro{QWI}{Quantum Well Intermixing}
 \acro{QPSK}{Quadrature Phase Shift Keying}
 \acro{RIE}{Reactive Ion Etching}
 \acro{ROADM}{Reconfigurable Optical Add-Drop Multiplexer}
 \acro{RMS}{Root Mean Square}
 \acro{SAG}{Selective-Area Growth}
 \acro{SEM}{Scanning Electron Microscope}
 \acro{SMF}{Single-Mode Fiber}
 \acro{SOA}{Semiconductor Optical Amplifier}
 \acro{SOI}{Silicon-on-Insulator}
 \acro{SSC}{Spot-Size Converter}
 \acro{TE}[$\mathrm{TE}$]{Transverse Electric\acroextra{ (The polarization with electric vector normal to the incidence plane. Equivalent to \emph{s}-polarization.)}}
 \acro{TM}[$\mathrm{TM}$]{Transverse Magnetic\acroextra{ (The polarization with magnetic vector normal to the incidence plane. Equivalent to \emph{p}-polarization.)}}
 \acro{TG}{Twin-Waveguide}
 \acro{TDM}{Time-Domain Multiplexing}
 \acro{TMAH}{TetraMethyl Ammonium Hydroxide}
 \acro{VC}{Vertical Coupler}
 \acro{VLSI}{Very Large Scale Integration}
 \acro{VWM}{Vertical Wavelength (De)Multiplexer}
 \acro{WDM}{Wavelength Division Multiplexing}
 \acro{WPD}{Waveguide Photodetector}
 \acro{WS}{Wavelength Splitter}
 \acro{XRD}{X-ray Diffraction}
%%%%%%%%%%%%%%Chemical Materials Begins%%%%%%%%%%%%%%%%%%%%%%%
 \acro{AlGaAs}[$\mathrm{AlGaAs}$]{Aluminium Gallium Arsenide}
 \acro{InP}[$\mathrm{InP}$]{Indium Phosphide}
 \acro{InGaAsP}[$\mathrm{InGaAsP}$]{Indium Gallium Arsenide Phosphide}
 \acro{GaAs}[$\mathrm{GaAs}$]{Gallium Arsenide}
 \acro{SiO2}[$\mathrm{SiO_2}$]{Silicon Dioxide\acroextra{ (silica)}}
 \acro{Si}[$\mathrm{Si}$]{Silicon}
%%%%%%%%%%%%%%Chemical Materials Finishes%%%%%%%%%%%%%%%%%%%%
\end{acronym}

\addcontentsline{toc}{chapter}{List of Acronyms}
\newpage

% Pages of List of Symbols
%You can create it manually in a file ListSymbols and then
%
\chapter*{List of Symbols}
%20041217 by Henghua Deng
%Chapter style and no numbering, not included in the Table of Contents (TOC)
%For smaller fonts: \begin{center} \Large \textbf{List of Acronyms} \end{center}

%NOTE that the environments of {list, itemize, description, enumerate}
% are not good for adjusting the spaces between an item and its description;
%%And environment {tabular} can not span two pages.
%%So environment {tabbing} is the best choice which adjusts spaces by tabs.
%
%\begin{tabbing} starts a columnar environment.
%Use commands \= (set tab), \> (tab), \< (backtab), \+ (indent one
%tab stop), \- (outdent one tab stop), \` (flush right), \' (flush
%left), \pushtabs, \poptabs, \kill, \\.
%
%See Latex123.pdf by Edward G.J. Lee (Taiwan) and Latex209CommandSummary.pdf
%
%%Creat a List of Symbols using package {nomencl} (DHH 20041217) See FrontPagesDHH.tex.

\begin{tabbing}
%XXXXXXXXXXXXX\=XXXXXXXXXXXXXXX \kill %\kill means do not show this row; %Or USE:
  \hspace{6em} \= \hspace{20em} \kill
  $t$ \> time \\
  ($i, j$) \> complex symbol ($ = \sqrt{-1}$) \\
  ($x, y, z$) \> Cartesian coordinates ($x$-horizontal, $y$-vertical, $z$-longitudinal) \\
  ($\emph{\textbf{i}}_{x}$, $\emph{\textbf{i}}_{y}$, $\emph{\textbf{i}}_{z}$)
        \> unit vectors along ($x$, $y$, $z$) directions \\
  $\nabla$ \> del (nabla) operator \\
  {} \> (in Cartesian coordinates
              $= \emph{\textbf{i}}_{x} \frac{\displaystyle \partial}{\displaystyle \partial x}
              +\emph{\textbf{i}}_{y}\frac{\displaystyle \partial}{\displaystyle \partial y}
              +\emph{\textbf{i}}_{z}\frac{\displaystyle \partial}{\displaystyle \partial z}
              = \nabla_t+\emph{\textbf{i}}_{z}\frac{\displaystyle \partial}{\displaystyle \partial z}$) \\
  $\nabla_t$ \> transversal del operator (in Cartesian coordinates
              $= \emph{\textbf{i}}_{x} \frac{\displaystyle \partial}{\displaystyle \partial x}
              +\emph{\textbf{i}}_{y}\frac{\displaystyle \partial}{\displaystyle \partial y}$) \\
  $\nabla^2$ \> Laplacian operator (in Cartesian coordinates
              $= \frac{\displaystyle \partial^2}{\displaystyle \partial x^2}
              + \frac{\displaystyle \partial^2}{\displaystyle \partial y^2}
              + \frac{\displaystyle \partial^2}{\displaystyle \partial z^2}$) \\
  $\nabla,\; \nabla \cdot,\; \nabla \times $ \> grad, div, curl operations \\
  $\Delta$ \> differential or waveguide index-contrast \\
  ($\varepsilon$, $\mu$) \> (permittivity, permeability) of dielectric material
                ($\varepsilon= \varepsilon_0 \varepsilon_r$, $ \mu = \mu_0 \mu_r$) \\
  ($\varepsilon_r$, $\mu_r$) \> relative (permittivity, permeability) ($\mu_r=1$ for non-magnetic materials) \\
  $\varepsilon_0$ \> free space permittivity ($=8.854187817 \times 10^{-12}{\ }\textrm{F/m}$) \\
  $\mu_0$ \> free space permeability ($=4\pi\times10^{-7}{\ }\textrm{H/m}$) \\
  ($\sigma_{E}$, $\sigma_{H}$) \> electric and magnetic conductibility \\
  $Y_0$ \> free space admittance ($Y_0=\sqrt{\frac{\displaystyle \varepsilon_0}{\displaystyle \mu_0}}$) \\
  $Z_0$ \> intrinsic impedance of vacuum ($Z_0=\sqrt{\frac{\displaystyle \mu_0}{\displaystyle \varepsilon_0}}
           =376.7303134749689\Omega$) \\
  $n_{eff}$ \> effective index \\
  $N$ \> complex refractive index. $N = n + i \cdot k$. \\
  $n$ \> refractive index (real part of the complex refractive index) \\
  $k$ \> extinction coefficient (imaginary part of the complex refractive index)\\
  %{} \> A finite value of $k$ for a medium denotes the presence of absorption. \\
  ($p, m, d, 0$) \> subscripts to denote prism, metal, dielectric, and free space materials \\
  \emph{\textbf{E}} \> electric field \\
  \emph{\textbf{H}} \> magnetic field \\
  \emph{R} \> reflectance \\
  \emph{T} \> transmittance \\
  $c$ \> velocity of the light in vacuum ($=\frac{\displaystyle 1}
         {\sqrt{\displaystyle \mu_0\varepsilon_0}}=2.99792458 \times 10^{8} \, \textrm{m/s}$)\\
  $\nu$ \> velocity of the light in dielectric materials
         ($=\frac{\displaystyle 1}{\sqrt{\displaystyle \mu \varepsilon}}$)\\
  $f$ \> frequency \\
  $\omega$ \> angular frequency ($= 2 \pi f$)\\
  $\lambda$ \> wavelength \\
  $k$ \> wavenumber ($=2\pi/\lambda$); in free space denoted as $k_0$\\
  $\beta$ \> propagation constant ($=n_{eff}k_0$) \\
  $\eta$ \> optical admittance \\
  $\theta$ \> angle of light beam \\
  $\theta_b$ \> Brewster angle for \emph{p}-polarization (TM wave, total refraction) \\
  $\theta_c$ \> critical angle for total internal reflection \\
  $\hbar$ \> Planck constant ($=6.626075540 \times 10^{-34} \textrm{J sec}$)
\end{tabbing}

\newpage
%but this is tedious and error-prone. (See nomencl.pdf by Boris Veytsman)
%
%Create a List of Symbols by using the package {nomencl} (DHH 20041217):
%Add \usepackage{nomencl} and \makeglossary to the preamble,
%issue \nomenclature{}{} for each symbol to be included in nomenclature list,
%and \printglossary in the place you want to have your nomenclature list.
%
%BUT NEED TO Run this command line manually: (Thesis2004DHH is the file name you working on)
%(please enable makeidx before this)
% makeindex Thesis2004DHH.glo -s nomencl.ist -o Thesis2004DHH.gls
%
%
%%Change display name "Nomenclature" to "List of Symbols":
%\renewcommand\nomname{List of Symbols}
%\printglossary  %%print nomencl (list of symbols) making
%\newpage

% Change page numbering back to Arabic numerals
%%\setcounter{page}{1}
\pagenumbering{arabic}
%%%%%%%%%%%%%%%%%%%%%%%%%%%%%%%%%%%%%%%%%%%%%%%%%%%%%%%%%%%%%

% Title page, declaration, borrowers' page, abstract, acknowledgements,
% dedication, table of contents, list of tables, list of figures,
% and List of Acronyms (by input{ListAcronyms} 20041216)

%% Coding Rules
%% Created: 2015-05-15; Updated: 2015-05-15;
%% by Henghua DENG, hdeng@optixera.com

\resetdatestamp %Date Stamp--Only use when custom package datestamp.sty is used.
%\begin{CJK}{UTF8}{gkai}

\chapter{代码规范} \label{ChCodeRule}
%======================================================================
%Heading Settings:
\markboth{Chapter~\thechapter.~Coding~Rules}{} %\leftmark calls #1 parameter
%\markright{ } % new right header. Only used for two-side documents.

\pagestyle{fancy} \fancyhead[RO,RE]{}
\fancyhead[LE]{\MakeUppercase{\leftmark}}
\fancyhead[LO]{\MakeUppercase{\rightmark}}
\fancyfoot[C]{\thepage} %\fancyfoot[C]{\thepage}

良好的软件开发习惯和代码规范可以:便于团体协作开发,有效提高软件开发效率,增强软件代码质量,增强软件运行速度,提高软件的通用性可移植性,降低代码出错机率,去除冗余代码,减少软件维护时间和成本,等等。所以,这里大概总结几条代码规范建议,希望可以使我们的团队代码开发更有效率。

%======================================================================
\section{代码规范} \label{SectCodeRuleDeng}
%======================================================================
良好的软件开发习惯和代码规范可以:便于团体协作开发,有效提高软件开发效率,增强软件代码质量,增强软件运行速度,提高软件的通用性可移植性,降低代码出错机率,去除冗余代码,减少软件维护时间和成本,等等。所以,这里大概总结几条代码规范建议,希望可以使我们的团队代码开发更有效率。

1.	分类目录:
软件代码按功能分类和分子目录,相同或相似功能的子程序文件放在同一目录下。单个程序文件尽量只包含相同目标的子函数。

2.	注释说明:
每个文件头,每个子程序头,每个函数头应该有简短的功能目的的说明注释。子函数内的关键部分(结构、算法、界面、变量等等)要有简短的描述。注释不需要长,而是应该言简意赅、简明扼要。

3.	层次结构:
代码希望层次分明、缩进整齐、格式统一,以便调试检错和扩展。

4.	命名规范:
变量和函数的命名尽量规范化。尽量使用有涵义的简短命名。规范、简短、有涵义、不重复、易区分、易查找、格式统一、尽量归类。同一类型的变量和函数使用类似的变量名字,命名可以加简短相同前缀(比如file****,layer****,cell****),以便于管理和查找。

5.	精简扼要:
冗余代码即刻删除!所有过时代码,如果希望今后继续参考,可以专门归总在某个目录或某个文件中。但是过时和冗余代码一定不要留在最终代码中。

6.	速度效率:
代码和算法尽量有效率,注意计算速度。能够用向量和矩阵处尽量用,能够用循环尽量用,减少单个变量和单个操作。任何低效的代码,一经循环调用,对程序的速度影响可以是毁灭性的。

7.	用户体验:
用户体验至上!降低软件使用所需要的学习时间,减少软件操作所需要的操作次数。比如设定目标“天下没有难做的版图!”,让入门门槛尽量降低。如果你自己都觉得怎样用不方便,那么用户更加会觉得不方便。有人说过,软件代码设计增加一次click会减少10%的用户。

8.	检查优化:
写完的程序代码一定要检查和优化,你会经常发现原来你可以做得更好,代码可以更规范更有效。尽量把程序当作文章来写,而不是散漫的代码,那么你就自然会该写的写,不该写的不写,该花功夫的地方花功夫,不该花功夫的地方不浪费。

9.	交流沟通:
及时和团队成员交流沟通。有任何问题需要团体其他成员注意或者解决,可以点名要求回答。
我们可约定在需要检查的地方加入代码注释如下:        //CHECK!!

统一思路,我们一定可以做到最好!

%%\pagestyle{empty}
%%\cleardoublepage
%%to generates one blank page for the next chapter to be on an odd page


%----------------------------------------------------------------------
% MAIN BODY AND CHAPTERS
%----------------------------------------------------------------------
% Put the document title and page numbers in the header
%\pagestyle{myheadings}
% Put title on left & chapter heading goes on right by default
%plain--Just a plain page number; empty--empty heads and feet, no page numbers;
%headings--headings on each page; myheadings--specify with \markboth or \markright commands.
%fancy--fancy headings, must have \usepackage{fancyhdr}, example here:
%\pagestyle{fancy} \fancyhead[LE,RO]{\thepage}
%\fancyhead[LO]{\bfseries \MakeUppercase{\leftmark}} \fancyhead[RE]{\rightmark}
%\fancyfoot[C]{\thepage} %L:left, R:right, O:odd, E:even; H:head; F:foot;
%\rightmark, \leftmark, \thepage, \lhead{}, \chead, \rhead{}, \lfoot{}, \cfoot{}, \rfoot{}
%See McECEthesis.sty by Peter Kabal of McGill Univ.
%\markboth{SOI PR with FEM}{} %\leftmark calls #1 parameter

% Go to normal sized type
\normalsize

%----------------------------------------------------------------------
% Chapters
%----------------------------------------------------------------------
% Include your "sub" source files here (must have extension .tex but do NOT write file name type)
%%\include{./Ch_DesignCharts/Ch_DesignCharts} for files in a directory.
%\include{./Ch_Intro}
%\include{./Ch_Prncpl}
%\include{./Ch_InitPopu}
%\include{./Ch_Optimum}
%\include{./Ch_Conclusion}
%\include{./Ch_MMI/Ch_MMI_Intro}

%%----------------------------------------------------------------------
%% Appendices
%%----------------------------------------------------------------------
%% Designate with \appendix declaration which just changes numbering style from here on
\appendix
%\include{./Apdx_CodeScript}

%----------------------------------------------------------------------
% Bibliography
%----------------------------------------------------------------------
% If done using the BibTeX program, use
%\bibliographystyle{plain} % sorted alphabetically, labeled with numbers
%\bibliography{keylatex} % names file keylatex.bib as my bibliography file
% OR, do it "by hand" inside a "thebibliography" environment
%%%%%%%%%%%%%%%%%%%%%%%%%%%
\bibliographystyle{IEEEtran}  %{IEEEtran.bst} %{IEEEtranS.bst} %plain, unsrt, abbrv, alpha
%\renewcommand\bibname{References} %\bibname for book-classes; %or \refname for article-classes.
\addcontentsline{toc}{chapter}{\bibname}

%\bibliography{../IEEE/IEEEabrv,../Deng,../AWG,../BOOK,../SIMU,../MGVI,../PRW,../SOI,../SPR,../THz}
%\bibliography{IEEEabrv,./biblib/AWG}
%\bibliography{IEEEabrv,./biblib/MMI}
%DO NOT INCLUDE FILE TYPE NAME!! %%NO SPACES BETWEEN DIFFERENT FILE NAMES!!

%----------------------------------------------------------------------
% END MATERIAL
%----------------------------------------------------------------------
% Glossary
% You could use a \begin{description} ... \end{description} for this
%\include{glossary}

% Index
% Put a \makeindex command in the Preamble if you use MakeIndex program
% and put
% \printindex % here
% OR, do it "by hand" inside \begin{theindex} ... \end{theindex}
%----------------------------------------------------------------------

%----------------------------------------------------------------------
% Curriculum Vitae %Optional
%----------------------------------------------------------------------

%----------------------------------------------------------------------

\end{document}
