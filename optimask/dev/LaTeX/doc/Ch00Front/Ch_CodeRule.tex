%% Coding Rules
%% Created: 2015-05-15; Updated: 2015-05-15;
%% by Henghua DENG, hdeng@optixera.com

\resetdatestamp %Date Stamp--Only use when custom package datestamp.sty is used.
%\begin{CJK}{UTF8}{gkai}

\chapter{代码规范} \label{ChCodeRule}
%======================================================================
%Heading Settings:
\markboth{Chapter~\thechapter.~Coding~Rules}{} %\leftmark calls #1 parameter
%\markright{ } % new right header. Only used for two-side documents.

\pagestyle{fancy} \fancyhead[RO,RE]{}
\fancyhead[LE]{\MakeUppercase{\leftmark}}
\fancyhead[LO]{\MakeUppercase{\rightmark}}
\fancyfoot[C]{\thepage} %\fancyfoot[C]{\thepage}

良好的软件开发习惯和代码规范可以:便于团体协作开发,有效提高软件开发效率,增强软件代码质量,增强软件运行速度,提高软件的通用性可移植性,降低代码出错机率,去除冗余代码,减少软件维护时间和成本,等等。所以,这里大概总结几条代码规范建议,希望可以使我们的团队代码开发更有效率。

%======================================================================
\section{代码规范} \label{SectCodeRuleDeng}
%======================================================================
良好的软件开发习惯和代码规范可以:便于团体协作开发,有效提高软件开发效率,增强软件代码质量,增强软件运行速度,提高软件的通用性可移植性,降低代码出错机率,去除冗余代码,减少软件维护时间和成本,等等。所以,这里大概总结几条代码规范建议,希望可以使我们的团队代码开发更有效率。

1.	分类目录:
软件代码按功能分类和分子目录,相同或相似功能的子程序文件放在同一目录下。单个程序文件尽量只包含相同目标的子函数。

2.	注释说明:
每个文件头,每个子程序头,每个函数头应该有简短的功能目的的说明注释。子函数内的关键部分(结构、算法、界面、变量等等)要有简短的描述。注释不需要长,而是应该言简意赅、简明扼要。

3.	层次结构:
代码希望层次分明、缩进整齐、格式统一,以便调试检错和扩展。

4.	命名规范:
变量和函数的命名尽量规范化。尽量使用有涵义的简短命名。规范、简短、有涵义、不重复、易区分、易查找、格式统一、尽量归类。同一类型的变量和函数使用类似的变量名字,命名可以加简短相同前缀(比如file****,layer****,cell****),以便于管理和查找。

5.	精简扼要:
冗余代码即刻删除!所有过时代码,如果希望今后继续参考,可以专门归总在某个目录或某个文件中。但是过时和冗余代码一定不要留在最终代码中。

6.	速度效率:
代码和算法尽量有效率,注意计算速度。能够用向量和矩阵处尽量用,能够用循环尽量用,减少单个变量和单个操作。任何低效的代码,一经循环调用,对程序的速度影响可以是毁灭性的。

7.	用户体验:
用户体验至上!降低软件使用所需要的学习时间,减少软件操作所需要的操作次数。比如设定目标“天下没有难做的版图!”,让入门门槛尽量降低。如果你自己都觉得怎样用不方便,那么用户更加会觉得不方便。有人说过,软件代码设计增加一次click会减少10%的用户。

8.	检查优化:
写完的程序代码一定要检查和优化,你会经常发现原来你可以做得更好,代码可以更规范更有效。尽量把程序当作文章来写,而不是散漫的代码,那么你就自然会该写的写,不该写的不写,该花功夫的地方花功夫,不该花功夫的地方不浪费。

9.	交流沟通:
及时和团队成员交流沟通。有任何问题需要团体其他成员注意或者解决,可以点名要求回答。
我们可约定在需要检查的地方加入代码注释如下:        //CHECK!!

统一思路,我们一定可以做到最好!

%%\pagestyle{empty}
%%\cleardoublepage
%%to generates one blank page for the next chapter to be on an odd page
