%% Main UI of Optimask Layout Software
%% Created: 2017-05-15; Updated: 2017-07-15;
%% by Henghua DENG, hdeng@optixera.com

\resetdatestamp %Date Stamp--Only use when custom package datestamp.sty is used.

%\part{Optimask Layout Design} \label{PartMaskDesign}
%本部分介绍Optimask版图设计软件基本框架,具体功能实现,主要界面,命令行及编程输入等等。

\chapter{Optimask基本界面} \label{ChMaskMainUI}
%======================================================================
%Heading Settings:
\markboth{Chapter~\thechapter.~Design~Architecture}{} %\leftmark calls #1 parameter
%\markright{ } % new right header. Only used for two-side documents.

\pagestyle{fancy}
\fancyhead[RO,RE]{}
\fancyhead[LE]{\MakeUppercase{\leftmark}}
\fancyhead[LO]{\MakeUppercase{\rightmark}}
\fancyfoot[C]{\thepage}
%%\fancyfoot[L]{\today}

目前我们的基本框架已经定型如下。
各区域缺省设置如上图。用户可在设置界面关闭或者取消指定区域,且可自由移动、放大和缩小区域。
具体到每个子区:

%======================================================================
\section{菜单栏(Menu)} \label{SectMaskMenus}
%======================================================================
% \OptiXera\Develop\optixera\Docs\软件开发代码规范.txt Optixera版图软件开发进阶.docx 

菜单栏包括通常的文件输入输出,程序设置,版图绘制,浏览编辑,程序调试,辅助工具,视窗选择,使用帮助等类别功能。

\subsection{文件(File)} \label{SectMaskMenuFile} 
程序的第一步操作是文件操作。
试验创建、打开、关闭文件。
试验导入导出“GDSII”文件。实例GDSII文档在
\begin{verbatim} \optixera\optimask\src\File\GDSII_Test_File \end{verbatim} 目录和
\begin{verbatim} \klayout-0.24.8\testdata\gds \end{verbatim} 目录。
试验导入导出“OASIS”文件。如下图所示。实例GDSII文档在
\begin{verbatim} \klayout-0.24.8\testdata\oasis \end{verbatim} 目录。OASIS文档和GDSII文档可以互相转换。
\subsubsection{File --> Import GDSII }
\begin{enumerate}
	\item 测试单个构元,单层。单个构元有多重部件。
	\item 测试单个构元,但是多层。
	\item 测试多个构元,而且多层。
\end{enumerate}

\subsection{编辑(Edit)} \label{SectMaskMenuEdit} 
\subsection{视图(View)} \label{SectMaskMenuView} 
\subsection{绘制(Draw)} \label{SectMaskMenuDraw} 

%i.	Polygon, Circle, Wire(用户可简单通过界面画图,鼠标或键盘。这是目前软件都可以做到的基本功能)
%ii.	Input by Matrix, or calculate polygon points.(这是目前所有版图软件无法独立做到的地方。) 
%iii.	扫描图形Scan Picture(这是目前版图软件比较难做到的地方)。也可扫描条形码,和二维码。算法存下的文件必须精确,且文件尺寸小。
%iv.	图形库(我们慢慢建立)。
%v.	字库。可引入任意字库(C:\Windows\Fonts)字体如上图例(而非限于仅有几个难看的字)。 

\subsection{变化(Alter)} \label{SectMaskMenuAlter} 
\subsection{构元(Cell)} \label{SectMaskMenuCell}
\subsection{构层(Layer)} \label{SectMaskMenuLayer}
\subsection{编程(Script, Code, Macro, Programming)} \label{SectMaskMenuCode} 
\subsection{配置(Config)} \label{SectMaskMenuCnfg} 
\subsection{工具(Tool)} \label{SectMaskMenuTool}
\subsection{窗口(Window)} \label{SectMaskMenuWndw} 
\subsection{帮助(Help)} \label{SectMaskMenuHelp}

%======================================================================
\section{工具条(Toolbar)} \label{SectMaskToolbar}
%======================================================================
工具条主要是对应菜单栏的所有功能。需要每个功能有对应的图标。每一个菜单栏所有的功能集中在同一个工具条上。

%======================================================================
\section{构层面板(Layer Panel)和构层配置(Layer Palette)} \label{SectMaskLayers}
%======================================================================
构层区和层设置区(Layers Palette)缺省随主界面,但是允许用户自由移动和关闭。(当层设置好后,有时用户需要大窗口进行绘图)

\subsection{构层面板(Layer Panel)} \label{SectMaskLayerPanel} 
构层面板(Layer Panel)主要是构层的列表。
其中表头分别为: Group, LAYER, Lock, Hide, Protect,Fill, GDSII Number(\#), GDSII Data Type (DT),Note。
表格数据为版图文件所真实包含的构层列表。对于每一构层,如果有对应表头的数据,那么就显示;如果没有表头对应的数据项,那么就显示为空即可。上图显示的表格数据只是一个虚例,真正的显示需要根据版图文件对应显示。
表格数据每一行对应一个构层。关于表头具体到单项:
\subsubsection{构层组群(Layer Group)或类别(Category)或排序(Order)}
构层组群(Layer Group)或类别(Category)或排序(Order)可以方便用户对多层组织,比如上图的群A有Chip和Mark两层。当然用户可以不组群,此时用户也可以通过这一栏设定特定的值来进行排序。
这里举个K-Layout参考示例 。对层群的操作需要传递到其从属的所有构层。
\subsubsection{构层名称(Layer Name)}
构层名称(Layer Name)记录构层的名称(必须有)。如果用户不给构层命名,那么程序可以自动设置构层名称为GDSII\#对应(读入GDSII文件一定有GDSII\#)。 比如Layer\#005(如果GDSII\#为5)。当然也有另外一种情况,有构层名称,但没有GDSII\#,这时表示该层不会输出到GDSII文件格式中去。
\subsubsection{锁定(Lock)}
是否锁定(LOCK=1加锁, UNLOCK=0解锁)。锁定某构层表示该构层不会被修改,即使用户修改了该层的物件,也不会允许被保存文件(此时可以弹出警告框提示)。锁定可以防止对指定层的误操作。所有可以修改的构层必须在解锁状态。
\subsubsection{隐藏(Hide)}
是否隐藏(HIDE=1=UNSHOW隐藏, UNHIDE=0=SHOW显示)。隐藏某构层时,在工作窗口上不会显示该层的任何图像。构层被隐藏时,也同时不可以被选中(即,同时处于PROTect状态)。
\subsubsection{保护(PROTect)}
是否保护(PROTect=1=Unselectable不可选中, UNPROT=0=Selectable可选中)。
\subsubsection{填充(Fill)}
是否填充(FILL=1实心, UNFILL=0空心)。选定是该构层的图形全部实心显示,否则空心显示。
\subsubsection{GDSII 层号Number (\#)}
导入导出时对应的GDSII构层号码。导入GDSII文件一定有GDSII\#。如果有构层名称,但没有GDSII\#,这时表示该层不会输出到GDSII文件格式中去。
\subsubsection{GDSII 数据类型(Data Type,DT)}
GDSII数据类型。
\subsubsection{注释(Note)}
对构层的注释(如果有必要)。可以为空白。

\subsection{构层配置(Layer Palette)} \label{SectMaskLayerPalette} 
构层配置(色板,格纹,动画,样式,能见度)基本套用KLayout 。其中(Optimask :KLayout)名字对应为: Fill Color = Color, Frame Color = Frame color, Pattern = Stipple; 其他不变。
构层色板(Layer Palette)包括下面子部分:
\subsubsection{构层格纹(Stipple, Pattern)}
\subsubsection{构层动画(Animation)}
\subsubsection{构层样式(Style)}
\subsubsection{构层能见度(Visibility)}

%======================================================================
\section{构元结构面板(Cell Structure Panel)} \label{SectMaskCellDock}
%======================================================================
CellDock构元结构面板(Cell Structure Tree Panel), 构元组织(Cell Hierarchy),模块树和组织结构区。显示各结构之间的从属关系。双击特定某个构元时,打开一个新工作窗口来显示该构元图形。

%======================================================================
\section{版图主视图区(Work Panel)} \label{SectMaskWorkDock}
%======================================================================
WorkDock版图主视图区 ---- 主要版图设计工作区。结构的绘图,调用,几何运算,显示等等。允许多重工作区域(Workspace, Work Panel)。

%======================================================================
\section{视图导航区域 (Navigator, Aerial View)} \label{SectMaskNaviDock}
%======================================================================
NaviDock视图导航区域 (Navigator, Aerial View) (导航区,鸟瞰区,第二视图区,辅助视图区)--- 显示当前结构在整个Wafer组装之后的位置等等。鼠标选中对应的位置时,主工作区可以显示该地方的详尽图;反之亦然。

%======================================================================
\section{编程区域(Command, Script, Macro, Code, Programming)} \label{SectMaskCodeDock}
%======================================================================
CodeDock编程区域(Command, Script, Macro, Code, Programming)---编程产生结构,以及结构组合宏命令(比如Assembling)上。

%======================================================================
\section{信息输出栏(Information, Status, Result, Output)} \label{SectMaskInfoDock}
%======================================================================
InfoDock信息输出栏(Information, Status, Result, Output) --- 显示程序运行时必要的显示信息,输出结果等等。显示当前结构的统计信息,比如多少层,多少多边型,多少精度等等。

%======================================================================
\section{提示报警栏(Hint, Error, Warning, Issue)} \label{SectMaskHintDock}
%======================================================================
HintDock提示报警栏(Hint, Error, Warning, Issue)报错和提示区---如果程序出错,提示如何纠正。

%%\pagestyle{empty}
%%\cleardoublepage
%%to generates one blank page for the next chapter to be on an odd page
